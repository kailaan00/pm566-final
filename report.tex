% Options for packages loaded elsewhere
\PassOptionsToPackage{unicode}{hyperref}
\PassOptionsToPackage{hyphens}{url}
%
\documentclass[
]{article}
\usepackage{amsmath,amssymb}
\usepackage{lmodern}
\usepackage{iftex}
\ifPDFTeX
  \usepackage[T1]{fontenc}
  \usepackage[utf8]{inputenc}
  \usepackage{textcomp} % provide euro and other symbols
\else % if luatex or xetex
  \usepackage{unicode-math}
  \defaultfontfeatures{Scale=MatchLowercase}
  \defaultfontfeatures[\rmfamily]{Ligatures=TeX,Scale=1}
\fi
% Use upquote if available, for straight quotes in verbatim environments
\IfFileExists{upquote.sty}{\usepackage{upquote}}{}
\IfFileExists{microtype.sty}{% use microtype if available
  \usepackage[]{microtype}
  \UseMicrotypeSet[protrusion]{basicmath} % disable protrusion for tt fonts
}{}
\makeatletter
\@ifundefined{KOMAClassName}{% if non-KOMA class
  \IfFileExists{parskip.sty}{%
    \usepackage{parskip}
  }{% else
    \setlength{\parindent}{0pt}
    \setlength{\parskip}{6pt plus 2pt minus 1pt}}
}{% if KOMA class
  \KOMAoptions{parskip=half}}
\makeatother
\usepackage{xcolor}
\usepackage[margin=1in]{geometry}
\usepackage{longtable,booktabs,array}
\usepackage{calc} % for calculating minipage widths
% Correct order of tables after \paragraph or \subparagraph
\usepackage{etoolbox}
\makeatletter
\patchcmd\longtable{\par}{\if@noskipsec\mbox{}\fi\par}{}{}
\makeatother
% Allow footnotes in longtable head/foot
\IfFileExists{footnotehyper.sty}{\usepackage{footnotehyper}}{\usepackage{footnote}}
\makesavenoteenv{longtable}
\usepackage{graphicx}
\makeatletter
\def\maxwidth{\ifdim\Gin@nat@width>\linewidth\linewidth\else\Gin@nat@width\fi}
\def\maxheight{\ifdim\Gin@nat@height>\textheight\textheight\else\Gin@nat@height\fi}
\makeatother
% Scale images if necessary, so that they will not overflow the page
% margins by default, and it is still possible to overwrite the defaults
% using explicit options in \includegraphics[width, height, ...]{}
\setkeys{Gin}{width=\maxwidth,height=\maxheight,keepaspectratio}
% Set default figure placement to htbp
\makeatletter
\def\fps@figure{htbp}
\makeatother
\setlength{\emergencystretch}{3em} % prevent overfull lines
\providecommand{\tightlist}{%
  \setlength{\itemsep}{0pt}\setlength{\parskip}{0pt}}
\setcounter{secnumdepth}{-\maxdimen} % remove section numbering
\ifLuaTeX
  \usepackage{selnolig}  % disable illegal ligatures
\fi
\IfFileExists{bookmark.sty}{\usepackage{bookmark}}{\usepackage{hyperref}}
\IfFileExists{xurl.sty}{\usepackage{xurl}}{} % add URL line breaks if available
\urlstyle{same} % disable monospaced font for URLs
\hypersetup{
  pdftitle={Change in Types of Crime Before and During the COVID-19 Pandemic},
  pdfauthor={Kaila An},
  hidelinks,
  pdfcreator={LaTeX via pandoc}}

\title{Change in Types of Crime Before and During the COVID-19 Pandemic}
\author{Kaila An}
\date{2022-12-05}

\begin{document}
\maketitle

\hypertarget{introduction}{%
\subsection{Introduction}\label{introduction}}

The COVID-19 pandemic had changed the world as we knew it like no other
occurrence before. It changed people's lifestyles, their habits, and
overall created a ``new normal''. People's mindsets and priorities have
changed so drastically, that it would be interesting to note whether or
not it brought a change to criminal behavior and activity. Ever since
the true start of the pandemic in early 2020, we have seen high rates of
crime, like the widespread looting, racial hate crimes, and much more.
For this project, I chose to focus on Los Angeles City Data,
specifically, `Crime Data in LA from 2020 to present' and `Crime Data in
LA from 2010 to 2019'. These particular data sets document incidents of
crime, transcribed from crime reports. This data has been provided by
the Los Angeles Police Department. Based on these data sets, the
question I would like to address is:

How has the prevalence of certain types of crimes in Los Angeles changed
since the start of the COVID-19 pandemic?

This final project for PM 566 will examine how the prevalence of certain
crimes have changed in LA county as a result of the pandemic and will do
so through designing a number of interactive figures that will make it
easier to visualize crime instances and characteristics of certain
crimes that have taken place. It will also make it possible to detect
and analyze certain patterns and contingencies that may be apparent.

\hypertarget{methods}{%
\subsection{Methods}\label{methods}}

\begin{longtable}[]{@{}lr@{}}
\caption{Prevalence of Different Types of Crimes in LA County
2010-2019}\tabularnewline
\toprule()
Crm Cd Desc & n \\
\midrule()
\endfirsthead
\toprule()
Crm Cd Desc & n \\
\midrule()
\endhead
BATTERY - SIMPLE ASSAULT & 190569 \\
BURGLARY FROM VEHICLE & 162184 \\
VEHICLE - STOLEN & 159903 \\
THEFT PLAIN - PETTY (\$950 \& UNDER) & 149910 \\
BURGLARY & 147731 \\
\bottomrule()
\end{longtable}

\begin{longtable}[]{@{}lr@{}}
\caption{Prevalence of Different Types of Crimes in LA County
2020-Present}\tabularnewline
\toprule()
Crm Cd Desc & n \\
\midrule()
\endfirsthead
\toprule()
Crm Cd Desc & n \\
\midrule()
\endhead
VEHICLE - STOLEN & 63711 \\
BATTERY - SIMPLE ASSAULT & 46445 \\
BURGLARY FROM VEHICLE & 36911 \\
VANDALISM - FELONY (\$400 \& OVER, ALL CHURCH VANDALISMS) & 36870 \\
BURGLARY & 35539 \\
\bottomrule()
\end{longtable}

\hypertarget{results}{%
\subsection{Results}\label{results}}

\textbf{Crime Occurrence by Year}

This interactive bar chart shows the number of crimes (total) that took
place yearly from 2010-2022. We see that rates of crime have remained
relatively similar, but there is a spike in crime between 2020 to 2021
(during the COVID-19 pandemic). There seems to be significantly less
crime in 2022. This is not due to the fact that crime rates had
decreased. It is important to note that the year 2022 is not over yet.
It would be worthwhile to revisit this data set at the end of the year
and run the same analysis, in order to see the ending total of crime
occurrences for 2022.

\begin{center}\includegraphics[width=700px]{report_files/figure-latex/unnamed-chunk-6-1} \end{center}

\textbf{Vandalism (Felony)}

A new crime type that arose in the Top 5 of 2020-2022 (during the
COVID-19 pandemic) was Vandalism. This would be considered a felony for
offenses that were \$400 in damages and over and included all church
vandalisms. This is a scatterplot that represents occurrences of these
sorts of vandalisms throughout the years. There is a red line to
indicate the start of the COVID-19 pandemic, to make it easier to see
where a difference lies, if any. Vandalism occurrences were on the rise
since the start of the pandemic and reached a high in 2021. The sudden
drop in 2022 can also be attributed to the fact that the year 2022 is
not over yet.

\begin{center}\includegraphics[width=700px]{report_files/figure-latex/unnamed-chunk-7-1} \end{center}

\textbf{Stolen Vehicles}

The \#1 most common crime after the start of the COVID-19 pandemic was
stolen vehicles. This is a scatterplot that represents occurrences of
these sorts of vehicle theft throughout the years 2010-2022. There is a
red line to indicate the start of the COVID-19 pandemic, to make it
easier to see where a difference lies, if any. We see that vehicle theft
was experiencing a steep upwards trend since 2019 and reached an
all-time high in 2021. It seems as though many crime types experienced
all-time highs in 2021. The sudden drop in 2022 can also be attributed
to the fact that the year 2022 is not over yet.

\begin{center}\includegraphics[width=700px]{report_files/figure-latex/unnamed-chunk-8-1} \end{center}

\textbf{Top 5 Crimes by Area} \{.tabset\}

The following figure shows the differing numbers of occurrence of the
top 5 crimes for each area of LA county. There are two tabs that
represent the top 5 crimes for the areas for 2010-2019 and the same for
2020-present. Through this figure, we can see how certain types of
crimes may be more prevalent in certain areas than others and if there
were overall more or less crime occurrences in different areas of LA
county. One can also navigate whether or not trends of crime
significantly changed in these specific areas following the start of the
COVID-19 pandemic.

\hypertarget{section}{%
\paragraph{2010-19}\label{section}}

\begin{center}\includegraphics[width=700px]{report_files/figure-latex/unnamed-chunk-9-1} \end{center}

\hypertarget{present}{%
\paragraph{2020-Present}\label{present}}

\begin{center}\includegraphics[width=700px]{report_files/figure-latex/unnamed-chunk-10-1} \end{center}

\hypertarget{section-1}{%
\subsubsection*{}\label{section-1}}
\addcontentsline{toc}{subsubsection}{}

\hypertarget{conclusion-and-summary}{%
\subsection{Conclusion and Summary}\label{conclusion-and-summary}}

\end{document}
